\documentclass[12pt]{article}

\usepackage{mystyle}

\title{Hidden Markov Models}
\author{
Justin T. Chiu
}
\date{\today}


\begin{document}
\maketitle

\begin{abstract}
TODO
\end{abstract}

\section{Introduction}
TODO

\section{Problem Setup}
We apply hidden markov models (HMMs) to language modeling,
where we would like to model sentences $\bx_{0:T}$.
The generative process of an HMM is as follows:
\begin{enumerate}
\item Choose an initial state $z_0 \sim \Cat(\nu), \nu \in \R^K$
\item For each time $t \in \set{1, \ldots, T}$ choose a state
$z_t \mid z_{t-1} \sim \Cat(\theta_{z_{t-1}}), \theta \in \R^{K \times K}$
\item For each time $t \in \set{0, \ldots, T}$ choose a word
$x_t \mid z_t \sim \Cat(\phi_{z_t}), \phi \in \R^{K \times V}$.
\end{enumerate}

This gives the following joint distribution:
\begin{equation}
\log p_\theta(\bx_{0:T}, \bz_{0:T})
= \log p_\theta(x_0, z_0) + \sum_{t=1}^T \log p_\theta(x_t, z_t \mid z_{t-1})\\
\end{equation}

\section{Parameter estimation}
We maximize the evidence of the observed sentences
$\log p(\bx_{0:T}) = \log \sum_{\bz_{0:T}} p(\bx_{1:T}, \bz_{0:T})$
via gradient ascent.

\subsection{Gradient of the evidence}
Let $\psi_0(z_0, z_1) = \log p(\bx_{0:1}, \bz_{0:1})$ and
$\psi_t(z_{t}, z_{t+1}) = \log p(x_{t+1}, z_{t+1} \mid z_{t})$ for $t \in \set{1, \ldots, T-1}$.
Additionally, let $\oplus$ and $\otimes$ be addition and multiplication in the log semiring.
After conditioning on the observed $\bx_{0:T}$, we can express the evidence as the following:
$$A_x = \log p(\bx_{0:T}) = \bigoplus_{\bz_{0:T}}\bigotimes_{t=0}^{T-1} \psi_t(z_t, z_{t+1})$$
where $A_\bx$ is the clamped log partition function.

We show that the gradient of the log partition function with respect to the $\psi_t(z_t, z_{t+1})$
is the first moment (a general result for the cumulant generating function
of exponential family distributions). Given the value of the derivative
$\frac{\partial A_\bx}{\partial \psi_t(z_t, z_{t+1})}$,
we can then apply the chain rule to compute the total derivative with respect to
downstream parameters. For example, the gradient with respect to the transition matrix
of the HMM is given by
$$\frac{\partial A_\bx}{\partial \theta_{ij}}
= \sum_t \frac{\partial A_\bx}{\partial \psi_t(i,j)}
\frac{\partial \psi_t(i,j)}{\partial \theta_{ij}}$$

Recall that the derivative of logaddexp ($\oplus$ in the log semiring) is
$$\frac{\partial}{\partial x} x \oplus y
= \frac{\partial}{\partial x} \log e^x + e^ y
= \frac{e^x}{e^x + e^y}
= \exp(x - (x \oplus y))
,$$
while the derivative of addition ($\otimes$ in the log semiring) is
$$\frac{\partial}{\partial x} x \otimes y = 1.$$
The derivative of the clamped log partition function $A_\bx$ is given by
\begin{align*}
\frac{\partial A_\bx}{\partial \psi_t(i,j)}
&= \frac{\partial}{\partial \psi_t(i,j)} \bigoplus_{\bz_{0:T}}
    \bigotimes_{t'} \psi_{t'}(z_{t'}, z_{t'+1})\\
&= \frac{\partial}{\partial \psi_t(i,j)} \left(
        \left(\bigoplus_{\bz_{0:T}:z_t = i, z_{t+1} = j}
        \bigotimes_{t'} \psi_{t'}(z_{t'}, z_{t'+1})\right)
        \bigoplus
        \left(\bigoplus_{\bz_{0:T}:z_t \ne i, z_{t+1} \ne j}
        \bigotimes_{t'} \psi_{t'}(z_{t'}, z_{t'+1})\right)
    \right)\\
&= \exp\left(\left(\bigoplus_{\bz_{0:T}:z_t = i, z_{t+1} = j} 
    \bigotimes_{t'} \psi_{t'}(z_{t'}, z_{t'+1})\right)
    \frac{\partial}{\partial \psi_t(i,j)}\left(\bigotimes_{t'} \psi_{t'}(z_{t'}, z_{t'+1})\right)
    - A_\bx\right)\\
&= \exp\left(\left(\bigoplus_{\bz_{0:T}:z_t = i, z_{t+1} = j} 
    \bigotimes_{t'} \psi_{t'}(z_{t'}, z_{t'+1})\right)
    - A_\bx\right)
\end{align*} 
which is the edge marginal for $z_t, z_{t+1}$ obtained via the forward-backward algorithm.
The second equality is a result of the associativity of $\oplus$,
while the third and fourth equalities are applications of the derivatives
derived above.

\subsubsection{Very high training loss}
It is okay if the surrogate loss is a loose bound.
We proved gradient estimator is correct.
This just means that the ELBO under a pairwise approximation is loose.

\section{Cloned HMM (CHMM)}
Inspired by \citet{dedieu2019learning},
we introduce a sparse emission constraint that allows us to
efficiently compute derivatives in large state spaces.
We constrain each word $x$ to be emit only by $z \in \mcC_x \subset \mcZ$:
\begin{equation}
p(x \mid z) \propto \begin{cases}
\phi_{z,x} & z \in \mcC_x \\
0 & \textrm{otherwise}
\end{cases}
\end{equation}
When conditioning on 

\bibliographystyle{plainnat}
\bibliography{bib}

\end{document}
